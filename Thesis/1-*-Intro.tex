\section{Introduction}

In 2010 a team of researchers developed a generalized method for
creating platform independent programs (PIPs)\citep{Brumley2010}. A PIP
is a special sort of program which can be run on multiple different
computer architectures without modification\footnote{For a more formal
  definition a PIP is a string of bytecode $b$ such that for different
  machines $m_1$ and $m_2$, $b$ is a valid program if:
  \[m_1(b) \not = \bot \wedge m_2(b) \not =\bot\]}. Unlike shell scripts
or programs written for a portable interpreter a PIP doesn't require
another program to run or compile it; rather it runs as a native program
on multiple architectures with potentially different behaviour on each.

To find a PIP you would have to analyze the architecture manuals for
each architecture you wanted and find the instructions in each which
compiled to identical patterns of bytecode and use them to construct
your PIP. The approach taken by the authors was to find small PIPs with
a very specific form: do nothing then jump\footnote{Consider the
  following example (taken from the original paper). The dissassembly
  for the x86 architecture is shown above, and for the MIPS platform
  bellow. \[\underbrace{\overbrace{90}^{\text{NOP}}
                                \overbrace{eb20}^{\text{JMP}}
                                2a
                               }_{\text{NOP}} 
        \underbrace{90eb203a}_{\text{NOP}}
        \underbrace{24770104}_{\text{B}}\] The string is valid on both
  platforms and has similar behaviour on both despite jumping to
  different locations. It is a valid PIP for the x86 and MIPS platforms.}.
By ensuring each architecture jumped to a different point and that each
architecture didn't accidentally run into a region anouther architecture
jumped into they could construct PIPs for any arbitrary program by
splitting them up into blocks of instructions specific to each
architecture and connecting them with the small PIPs.

They go on to give in the paper a generalized algorithm for constructing
these PIPs, and say that they have a working implementation of it for
creating PIPs for the x86, ARM, and MIPS platforms, as well as the
Windows, Mac, and Linux operating systems.

\subsection{Aim of the Project}

For this thesis I have implemented a small section of the PIP generation
algorithm---finding the \emph{gadget headers}; the PIPs that link the
specific code sections together. To generate the PIPs a list of
\emph{semantic NOPS}\footnote{A semantic NOP is an instruction which has
  no effect, but which might not necessarily be the \emph{NOP} assembly
  instruction. For example the ARM instruction: \lstinline!MOV r4, r4!
  Causes the contents of register four to be moved into register four
  and as such is equivalent to an actual \lstinline!NOP! instruction.
  Equally the sequence of instructions: \lstinline!PUSH r3!
  \lstinline!POP r3! If equivalent to two \lstinline!NOP! instructions
  when taken as a whole and so is a \emph{multi-instruction semantic
  NOP}.} and potential branch instructions has been found for each
architecture in the original paper and to extend the work of the
original paper I have analyzed two new platforms: the Java Virtual
Machine, and XMOS's XS1.

Unfortunately there does not seem to be a public database of these
instructions available for any architectures. Semantic NOPs have been
used in areas other than creating PIPs, for example malware
classification\citep{Bilar2007}\citep{Preda2007}, but there still
appears to be no exhaustive list exists documenting them. Part of the
work required is to create one.

\subsection{Why is this interesting?}

PIPs can be used for a variety of applications. One potential
application suggested by Brumley et. al.\citep{Brumley2010} is for
sneaking programs past dynamic execution checkers. Suppose two prisoners
\emph{Alice} \& \emph{Bob}, wish to send a malicious program between
themselves. To send the message they have to send it through a
\emph{Warden} who checks first that there communications don't contain
anything malicious and only delivers the message if its believed to be
harmless.

To sneak the program Alice and Bob use a form of
\emph{keyed-steganography}\footnote{which the authors\citep{Brumley2010}
  call \emph{execution-based steganography}.}. The program which they
wish to communicate becomes their \emph{steg-text}, and they construct a
\emph{cover-text} by writing some other program which doesn't contain
anything malicious. They either pre-arrange a shared secret which is the
platform that their programs should really be run on: this forms the
key. With the cover-text and steg-text created they create their message
by generating them into a PIP where on one platform (x86 say) it appears
to run the innocuous program and on the secret key platform (ARM for the
sake of argument) it runs the program they really wish to communicate.
Alice hands the warden the program and tells him that it is for the
innocuous architecture\footnote{if they were using ELF they wouldn't
  even need to do that---it's part of the header in the
  file\citep{SysVElf}.}. The warden runs the program and sees it isn't
something he would want to restrict and delivers it. In fact unless he
is aware that it has been in constructed in this way he may not even
check any other architectures as for most platforms it will appear to be
garbage just like any normal executable.

Another application is \emph{exfiltration protection}\footnote{Exfiltration
  is military term meaning the removal of a resource from enemy control.
  In the context of PIPs were talking about taking programs from
  protected PCs; kind of like DRM.}. Here the idea is that to protect
its software from theft a secret agency could make a modification to an
existing platform (the JVM or another virtual machine would be a good
choice here) and compile their program for this modified platform. They
then create another program for the unmodified platform which does
something else; maybe it phones home, maybe it destroys itself from the
computer it's running on. They create a PIP out of these two programs
and now if the program is stolen and the exfiltrator isn't aware of the
PIP nature (or exactly what modifications were made to the architecture)
they're not going to be able to run the program they removed.

Microcode offers another neat way to use PIPs. Suppose an attacker
manages to compromise a system in such a way that they can alter the
microcode of the processor, such as the recent HP printer
attack\citep{PrintMe2011}. Now suppose that as well as the microcode
update they also modify an existing program\footnote{In the PIP
  paper\citep{Brumley2010} they suggest \lstinline!ls!.} so that on the
compromised system it gives a backdoor or acts maliciously, but on
another (say one which is trying to forensically work out what is wrong
with the printer) it acts normally. Brumley et. al. go on to point
out\citep{Brumley2010} that if this was done by Intel and the PIP was a
preexisting and digitally signed application then it is a particularly
scary prospect. Merely signing the program would be insufficient protect
a user it would not check if the machine it was executing on had been
modified.

Other applications of PIPs include shellcode and viruses. For shellcode
the idea is that you can write it once and use it anywhere. For viruses
the idea is that if you could get the virus on a disk that is mounted on
multiple architectures (say an NTFS share or USB key) then you can
attack any platform you're plugged into.

The problem here is that although PIPs could be used to write
architecture independant programs there are more elegant solutions
available than relying on the intersection of instruction sets between
architectures. There are a couple of preexisting systems for doing this
such as Apple's \emph{Universal Binary} or the
\emph{FatELF}\citep{FatELF} format.

\subsection{Challenge}

\subsection{Summary}

