\section{Literature Reviews}

\subsection{Dynamic Binary Translation and Optimization (Ebcioglu01)}

The aim of the paper is to describe a VLIW\footnote{VLIW (Very Long
  Instruction Word) is a CPU architecture designed such that it can take
  advantage of instruction level parallelism.} architecture that is
compatible with an existing architecture (PowerPC in this case). The new
architecture, DAISY, can execute existing programs through the use of
dynamic compilation.

Unlike the PIP aproach they design their architecture to be binary
compatible (or at least with an architecture level VM) with PPC, rather
than by relying on overlaps in the IS and platform specific behaviour.

Again, self modifying code is an issue; but less so on the PPC arch.
They admit it'd be a pain on x86.

\subsection{Binary Translation (Sites93)}

Paper describes various methods of translating programs from one
architecture to another without using virtualization. They discuss the
various methods for translating between the VAX and OpenVMS
architectures.

They point out that certain things can make programs untranslatable
namely: exception handling differences, undocumented features, and
system specific behavior (VAX memory management or program-image
formats).

They do analysis on their programs. They note on the MIPS architecture:

\begin{quote}
only 10 instructions account for about 85\% of all code.

\end{quote}
This will probably be useful when trying to detect my PIPs from
non-PIPs. They also add that certain sequences of instructions are
idiomatic on a given architecture: again this can probably be exploited
for distinguishing PIPs.

This seems to be more to do with the ∏-translation step of the original
PIP paper, rather than the actual design of steganographic execution.
However it might be interesting with the dual problem: given a program
$p$ for a machine $M_1$ such that $M_1(p) = b_1$ find $M_2$ and $b_2$
such that $M_2(p) = b_2$.

\subsection{English Shellcode}

The paper describes a method of hiding shellcode such that it is hidden
inside English text---making the detection of it harder. They take
advantage of grammar and NL techniques to obscure the code.

Currently shellcode attacks can be spotted at compile time with linters
(e.g.~RATS. See also AEG paper), and through an emulation layer, taint
analysis, or by checking inputs against known shellcodes.

\begin{quote}
User input or network traffic is considered suspicious when it is
executable or anomalous

\end{quote}
This is \emph{insanely} cool. Hide a PIP gadget inside a buffer such
that it looks like its holding some text. Looks completely (well
relatively) inoccuous but then you're executing it.

\subsection{Automatic Exploit Generation}

The AEG challenge: given a program find vulnerabilities and generate
exploits automatically.\\Managed to find two new zero-days. Uses
heuristics to rank potentially executable areas \& to search them; to
look first at HTTP get requests for example.

Requires the source code (its a linter). They have a nice video showing
it working on a bug in \texttt{iwconfig} involving a \texttt{strcpy()}
call.

An extension/reworking of the LLVM projects KLEE symbolic execution
engine. Works better for this purpose but most exploits found usually
benefit from a little hand tweaking first.

Can prioritise search areas based on different statistics (i.e.~if a
programmer is known to make off-by-one errors they can prioritize buffer
error searching). Currently able to produce \emph{return-to-stack} and
\emph{return-to-libc} attacks.

\subsubsection{Return-to-stack attack}

Change the return address of a function so that we return to somewhere
on the stack where our shellcode has been placed.

\subsubsection{Return-to-libc}

Find a `\texttt{/bin/sh}' string in an executable (difficult!), then
overwrite a return address to jump into an `\texttt{execve /bin/sh}'
call.

Both these attacks still need to hijack control flow though. Neither
attack describes how to do this, but once you have managed to get a
shell it is effectively game over.

Very cool, but not what I'm really trying to do.

\subsection{Design of a Resourcable and Retargetable Binary Translator}

Automatic translation of executables from one format to another without
the source code (and by the \emph{NoWeb} guy!). Essentially they do it
by reverse engineering and decompiling before recompiling it.

Only works well on EXE formats at the moment. Uses an intermediate
language, but it seems very early. Small programs with negligable
overhead.

\subsection{An Information-Theoretic Model for Steganography}

