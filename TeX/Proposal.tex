\documentclass{article}
\usepackage{jahstyle}

\begin{document}

\section{Project Proposal}

\subsection{Title}

Platform Independent Programs---Steganographic Execution

\subsection{Advisor}

Dr.~Dan Page

\subsection{Motivation}

Until recently it was believed that having a program that ran on
multiple architectures was limited to virtual machines, and a few hand
crafted examples of bytecode that would execute on multiple platforms.
In 2010, however, a team of researchers from Carnegie Mellon University
succeeded in developing a method for creating multi-architecture
programs automatically for the x86, ARM and MIPS platforms, as well as
between the Darwin, FreeBSD and Linux operating systems.

At the moment there are no publicly available tools to create the
platform independent programs, and there has been little work on
detecting a PIP from a program for a single architecture. There haven't
been any attempts to extend the Carnegie Mellon teams approach to other
architectures (such as XMOS's XS1 or the JVM), and there aren't any
databases detailing what the gadget headers (which are the key to
developing platform independent execution) are between platforms.

There are many applications of platform independent execution. It can be
used as a keyed steganographic system for resisting attempts to steal
software. Alternatively it can be used to smuggle programs past a warden
in the classic \emph{Prisoners' Problem}. It can be used to create
generic shellcode that will run on multiple architectures, and it can be
used to create viruses that target specific computers on a network
similar to the Stuxnet worm. They also offer an alternative to the
\emph{Universal Binary} or \emph{FatELF} formats for executables, and
can be integrated with the pre-existing \emph{COFF} executable format
with no changes.

\subsection{Objectives}

\begin{itemize}
\item
  Implement the existing PIP-shellcode generation algorithm using a
  better-than-brute-force method to find the bytecode patterns that
  define a gadget header.
\item
  Attempt to implement the single instruction PIP generation algorithm,
  by extending the PIP-shellcode algorithm from the whole program level
  to smaller subsections of code.
\item
  Extend the PIP generation algorithm to a new architecture, by finding
  the intersection of semantic-NOPs and jumps between different
  instruction sets and using HPC resources.
\item
  Investigate methods to detect PIP from non-PIP by analyzing
  instruction density, searching for possible gadget headers under
  certain architectures, and using static analysis techniques to analyze
  code coverage within an executable.
\end{itemize}
\subsection{Plan}

\begin{itemize}
\item
  Study the architecture manuals for any relevant architectures. Find
  all semantic-NOPs and semantic-JUMPs in each architecture and build a
  templating system to generate gadget headers. Run templating system to
  generate a database of gadget headers. (A long time)
\item
  Implement the RG-generation algorithm for PIPs at the whole program
  level, and then the single instruction level. (Hopefully less time)
\item
  Use static analysis to find distinguishing characteristics of PIPs and
  develop an oracle for distinguishing PIP from non-PIP with a success
  rate better than guessing. (Depends\ldots{} Probably a fair amount of
  time).
\end{itemize}

\end{document}
